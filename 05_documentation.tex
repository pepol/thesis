\chapter{Dokument\'{a}cia syst\'{e}mu}
\label{ch:doc}

V tejto kapitole popíšeme inštaláciu, konfiguráciu a používanie systému. Taktiež
popíšeme formát BUILD súborov a prácu s nimi. Nakoniec vysvetlíme postup vývoja
nových typov cieľov a akcií.

\section{In\v{s}tal\'{a}cia}
\label{sec:installation}

\subsection{Prerekvizity}
\label{sec:installation:dependencies}

Aplikácia nepotrebuje pre svoj beh žiadnu externú aplikáciu, kedže jazyk Go linkuje
spustiteľné súbory staticky.

Pre inštaláciu zo zdrojových súborov je potrebovaný kompilátor jazyka Go vo verzii
aspoň 1.4.2. Návod na inštaláciu je na stránke \url{https://golang.org/dl/}. Taktiež
program potrebuje knižnicu simpleyaml. Je dostupná na stiahnutie na stránke
\url{https://github.com/smallfish/simpleyaml/} alebo pomocou príkazu:
\verb!go get github.com/smallfish/simpleyaml!.

Zdrojový kód aplikácie sa nachádza na stránke \url{https://github.com/imterra/forge},
získať sa dá napríklad pomocou príkazu \verb!git clone https://github.com/imterra/forge.git!.

\subsection{Master}
\label{sec:installation:master}

Zdrojové kódy Master komponenta sa nachádzajú v priečinku \texttt{client/}. Pre jeho
kompiláciu potrebujeme v tomto priečinku spustiť príkaz \verb!go build!. Tento príkaz
vytvorí binárny spustiteľný súbor \texttt{client}, ktorý nainštalujeme nakopírovaním
do niektorého z priečinkov v premennej \texttt{PATH} a premenovaním na \texttt{forge}.

\subsection{Slave}
\label{sec:installation:slave}

Inštalácia Slave komponentu je ekvivalentná k inštalácii Master komponentu, ktorú
sme popísali v časti~\ref{sec:installation:master}.

Slave komponent sa nachádza v priečinku \texttt{worker/} a jeho kompiláciu dosiahneme
spustením príkazu \verb!go build!. Tento príkaz vytvorí spustiteľný binárny súbor
\texttt{worker}, ktorý prekopírujeme do priečinku z premennej \texttt{PATH} a premenujeme
na \texttt{forge-server}.

\section{Konfigur\'{a}cia}
\label{sec:configuration}

V tejto časti popisujeme spôsoby nastavenia oboch komponentov. Ako prvý uvádzame Slave
komponent, kedže Master vyžaduje aspoň jeden bežiaci Slave komponent na svoju činnosť.

\subsection{Slave}
\label{sec:configuration:slave}

Slave komponent nemá vo verzii 1.0 implementované nastavenia cez konfiguračný
súbor. Všetky nastavenia sa určujú pri spustení, cez štandardné argumenty príkazového
riadka.

Každý z argumentov sa pri volaní definuje vo forme \texttt{--nazov=hodnota}, kde
názov je meno argumentu a hodnota je hodnota ktorú mu chceme špecifikovať.

\subsubsection{jobs}

Určuje maximálny počet súčasne bežiacich kompilačných procesov. Prednastavená hodnota
je rovnaká ako počet CPU jadier na počítači.

\subsubsection{port}

Určuje TCP port, na ktorom Slave prijíma volania RPC služby pre vykonanie úloh.
Táto hodnota je potrebná pri definovaní adresy Slave komponentu na strane Mastera.
Prednastavená hodnota je 1103.

\subsubsection{root}

Určuje koreňovú zložku projektov na serveri. V rámci tejto zložky sa ukladajú všetky
súbory ktoré Slave prijme, alebo vytvorí. Pokiaľ by hodnota argumentu nebola nastavená,
použije sa hodnota premennej prostredia \texttt{FORGE\_ROOT}. V prípade, že ani premenná
prostredia nie je nastavená, prednastavenou hodnotou je priečinok \texttt{forge/} v domovskom
adresári používateľa, ktorý Slave komponent spustil. Priečinok takto určený musí existovať.

\subsection{Master}
\label{sec:configuration:master}

Master komponent používa konfiguračný súbor vo formáte YAML, ktorý sa, pokiaľ
nie je povedané inak, nazýva \texttt{.forge.yaml} a nachádza sa v domovskom adresári
používateľa spúšťajúceho Master komponent. V prípade jeho neexistencie sa pokúsime
získať nastavenia z globálneho súboru \texttt{/etc/forge.yaml}. Konfiguračný súbor
nie je vyžadovaný na úspešné spustenie aplikácie.

Tento súbor môže špecifikovať rovnaké hodnoty ako argumenty na príkazovom riadku, ktoré
sú mu však nadradené.

\subsubsection{jobs}

Toto nastavenie špecifikuje počet súčasne bežiacich úloh pre lokálne spustený
Slave komponent. Pokiaľ je toto číslo nastavené na 0, lokálny komponent sa nespúšťa.
Prednastavenou hodnotou je počet jadier CPU na lokálnom počítači.

\subsubsection{root}

Určuje koreňovú zložku projektov na serveri. Všetky súbory projektov vyžadované
na kompiláciu sa musia nachádzať v tejto zložke a ich cesty musia byt v BUILD súboroch
špecifikované relatívne k nej. V prípade nešpecifikovania tohto argumentu na príkazovom
riadku sa použije premenná prostredia \texttt{FORGE\_ROOT}, v prípade jej neexistencie
sa použije nastavenie z konfiguračných súborov. V prípade, neexistencie konfiguračných
súborov sa ako koreňová zložka použije priečinok \texttt{.forge/} z domovského
adresára používateľa.

\subsubsection{worker}

Určuje adresy Slave komponentov (okrem lokálneho ak sa tento spúšťa automaticky), ktoré
sa majú použiť pri kompilácii. Adresy sa uvádzajú v štandardnom tvare \texttt{adresa:port}, kde
adresa je buď IP alebo DNS adresa servera.

Na príkazovom riadku môže byť viac Slave komponentov uvedených použitím niekoľkých argumentov
worker, alebo ako čiarkami oddelený zoznam pre jeden argument.

\begin{verbatim}
forge --worker=slave1.workers.com:1103,slave2.workers.com:1103
forge --worker=slave1.workers.com:1103 --worker=slave2.workers.com:1103
\end{verbatim}

V prípade konfiguračného súboru sa zoznam Slave komponentov určuje ako
zoznam jazyka YAML pod hlavičkou \texttt{worker}.

\section{BUILD s\'{u}bory}
\label{sec:buildfiles}

\section{Prid\'{a}vanie nov\'{y}ch cie\v{l}ov a akci\'{i}}
\label{sec:contributing}
