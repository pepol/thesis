\chapter{Dokument\'{a}cia syst\'{e}mu}
\label{ch:doc}

V tejto kapitole popíšeme inštaláciu, konfiguráciu a používanie systému. Taktiež
popíšeme formát BUILD súborov a prácu s nimi. Nakoniec vysvetlíme postup vývoja
nových typov cieľov a akcií.

\section{In\v{s}tal\'{a}cia}
\label{sec:installation}

\subsection{Prerekvizity}
\label{sec:installation:dependencies}

Aplikácia nepotrebuje pre svoj beh žiadnu externú aplikáciu, kedže jazyk Go linkuje
spustiteľné súbory staticky.

Pre inštaláciu zo zdrojových súborov je potrebovaný kompilátor jazyka Go vo verzii
aspoň 1.4.2. Návod na inštaláciu je na stránke \url{https://golang.org/dl/}. Taktiež
program potrebuje knižnicu simpleyaml. Je dostupná na stiahnutie na stránke
\url{https://github.com/smallfish/simpleyaml/} alebo pomocou príkazu: \\
\verb!go get github.com/smallfish/simpleyaml!.

Zdrojový kód aplikácie sa nachádza na stránke \url{https://github.com/imterra/forge},
získať sa dá napríklad pomocou príkazu \\
  \verb!git clone https://github.com/imterra/forge.git!.

\subsection{Master}
\label{sec:installation:master}

Zdrojové kódy Master komponenta sa nachádzajú v priečinku \texttt{client/}. Pre jeho
kompiláciu potrebujeme v tomto priečinku spustiť príkaz \verb!go build!. Tento príkaz
vytvorí binárny spustiteľný súbor \texttt{client}, ktorý nainštalujeme nakopírovaním
do niektorého z priečinkov v premennej \texttt{PATH} a premenovaním na \texttt{forge}.

\subsection{Slave}
\label{sec:installation:slave}

Inštalácia Slave komponentu je ekvivalentná k inštalácii Master komponentu, ktorú
sme popísali v časti~\ref{sec:installation:master}.

Slave komponent sa nachádza v priečinku \texttt{worker/} a jeho kompiláciu dosiahneme
spustením príkazu \verb!go build!. Tento príkaz vytvorí spustiteľný binárny súbor
\texttt{worker}, ktorý prekopírujeme do priečinku z premennej \texttt{PATH} a premenujeme
na \texttt{forge-server}.

\section{Konfigur\'{a}cia}
\label{sec:configuration}

V tejto časti popisujeme spôsoby nastavenia oboch komponentov. Ako prvý uvádzame Slave
komponent, kedže Master vyžaduje aspoň jeden bežiaci Slave komponent na svoju činnosť.

\subsection{Slave}
\label{sec:configuration:slave}

Slave komponent nemá vo verzii 1.0 implementované nastavenia cez konfiguračný
súbor. Všetky nastavenia sa určujú pri spustení, cez štandardné argumenty príkazového
riadka.

Každý z argumentov sa pri volaní definuje vo forme \texttt{--nazov=hodnota}, kde
názov je meno argumentu a hodnota je hodnota ktorú mu chceme špecifikovať.

\subsubsection{jobs}

Určuje maximálny počet súčasne bežiacich kompilačných procesov. Prednastavená hodnota
je rovnaká ako počet CPU jadier na počítači.

\subsubsection{port}

Určuje TCP port, na ktorom Slave prijíma volania RPC služby pre vykonanie úloh.
Táto hodnota je potrebná pri definovaní adresy Slave komponentu na strane Mastera.
Prednastavená hodnota je 1103.

\subsubsection{root}

Určuje koreňovú zložku projektov na serveri. V rámci tejto zložky sa ukladajú všetky
súbory ktoré Slave prijme, alebo vytvorí. Pokiaľ by hodnota argumentu nebola nastavená,
použije sa hodnota premennej prostredia \texttt{FORGE\_ROOT}. V prípade, že ani premenná
prostredia nie je nastavená, prednastavenou hodnotou je priečinok \texttt{forge/} v domovskom
adresári používateľa, ktorý Slave komponent spustil. Priečinok takto určený musí existovať.

\subsection{Master}
\label{sec:configuration:master}

Master komponent používa konfiguračný súbor vo formáte YAML, ktorý sa, pokiaľ
nie je povedané inak, nazýva \texttt{.forge.yaml} a nachádza sa v domovskom adresári
používateľa spúšťajúceho Master komponent. V prípade jeho neexistencie sa pokúsime
získať nastavenia z globálneho súboru \texttt{/etc/forge.yaml}. Konfiguračný súbor
nie je vyžadovaný na úspešné spustenie aplikácie.

Tento súbor môže špecifikovať rovnaké hodnoty ako argumenty na príkazovom riadku, ktoré
sú mu však nadradené.

\subsubsection{jobs}

Toto nastavenie špecifikuje počet súčasne bežiacich úloh pre lokálne spustený
Slave komponent. Pokiaľ je toto číslo nastavené na 0, lokálny komponent sa nespúšťa.
Prednastavenou hodnotou je počet jadier CPU na lokálnom počítači.

\subsubsection{root}

Určuje koreňovú zložku projektov na serveri. Všetky súbory projektov vyžadované
na kompiláciu sa musia nachádzať v tejto zložke a ich cesty musia byt v BUILD súboroch
špecifikované relatívne k nej. V prípade nešpecifikovania tohto argumentu na príkazovom
riadku sa použije premenná prostredia \texttt{FORGE\_ROOT}, v prípade jej neexistencie
sa použije nastavenie z konfiguračných súborov. V prípade, neexistencie konfiguračných
súborov sa ako koreňová zložka použije priečinok \texttt{.forge/} z domovského
adresára používateľa.

\subsubsection{worker}

Určuje adresy Slave komponentov (okrem lokálneho ak sa tento spúšťa automaticky), ktoré
sa majú použiť pri kompilácii. Adresy sa uvádzajú v štandardnom tvare \texttt{adresa:port}, kde
adresa je buď IP alebo DNS adresa servera.

Na príkazovom riadku môže byť viac Slave komponentov uvedených použitím niekoľkých argumentov
worker, alebo ako čiarkami oddelený zoznam pre jeden argument.

\begin{verbatim}
forge --worker=slave1.workers.com:1103,slave2.workers.com:1103
forge --worker=slave1.workers.com:1103 --worker=slave2.workers.com:1103
\end{verbatim}

V prípade konfiguračného súboru sa zoznam Slave komponentov určuje ako
zoznam jazyka YAML pod hlavičkou \texttt{worker}.

\section{BUILD s\'{u}bory}
\label{sec:buildfiles}

BUILD súbory popisujú vzťahy medzi súbormi v projekte ako sériu cieľov, ktorých
splnením sa splní požadovaná úloha, najčastejšie kompilácia. Ako formát týchto
súborov sme použili formát YAML, ktorý sme popísali v časti~\ref{sec:yaml}.

\begin{listing}
  \inputminted[frame=lines, framesep=2mm,linenos,fontsize=\scriptsize,firstline=1,lastline=13]{yaml}{/home/pepol/src/auvm/build.yaml}
  \caption{Ukážka BUILD súboru projektu auvm}
  \label{lst:buildfile}
\end{listing}

Na listingu~\ref{lst:buildfile} sme uviedli príklad BUILD súboru projektu auvm\footnote{\url{https://github.com/pepol/auvm}}.

\subsection{Defin\'{i}cia cie\v{l}a}
\label{sec:buildfiles:targetdef}

Definíciou cieľa je asociatívne pole, ktorého kľúče a s nimi asociované hodnoty, spoločne
s názvom cieľa, presne popisujú každý cieľ. Teraz popíšeme všetky povolené kľúče.

\subsubsection{type}

Jediná informácia vyžadovaná v akomkoľvek cieli, popisuje výsledky aj podmienky pre
splnenie daného cieľa. V súčasnej verzii podporujeme ciele typu \texttt{lib\_c} pre
knižnice jazyka C a \texttt{app\_c} pre spustiteľné aplikácie v jazyku C.

Výsledkom splnenia cieľa typu \texttt{lib\_c} je súbor s názvom \texttt{nazovcieľa.a}
obsahujúci staticky linkované objektové súbory, ktoré vzniknú kompiláciou zdrojových
súborov.

Po splnení cieľa s typom \texttt{app\_c} je spustiteľný súbor s rovnakým názvom
ako je názov cieľa. Tento súbor vznikne zlinkovaním knižníc špecifikovaných
ako závislosti a skompilovaných zdrojových súborov.

\subsubsection{sources, resources}

Hodnoty \texttt{sources} a \texttt{resources} sú polia jazyka YAML obsahujúce
za sebou vymenované mená súborov. Povolené a podporované je určenie súborov v adresári
BUILD súboru alebo jeho podadresároch pomocou určenia cesty relatívnej k polohe
BUILD súboru.

V prípade typu cieľa v jazyku C sa každý súbor z poľa \texttt{sources} samostatne skompiluje
do objektového súboru, následne sú všetky tieto súbory zlinkované do statickej knižnice a, v prípade
typu aplikácie, následne zlinkované s výstupmi cieľov, na ktorých tento cieľ závisí, vytvárajúc
spustiteľnú aplikáciu.

Pre ciele v jazyku C sa súbory z poľa \texttt{resources} používajú na určenie nekompilovaných
súborov, ktoré sú ale ku kompilácii potrebné. Jedná sa napríklad o hlavičkové súbory jazyka
C. Súbory v tomto poli sú totižto iba prenesené na kompilujúci počítač, prípadne pri
opätovnej kompilácii je aj ich obsah kontrolovaný voči zmenám.

\subsubsection{dependencies}

Pole \texttt{dependencies} obsahujé názvy cieľov, ktorých splnenie je predpokladom
k úspešnému splneniu aktuálneho cieľa. Meno cieľa môže byť uvedené buď ako meno cieľa
z rovnakého BUILD súboru, alebo ako meno cieľa definovaného v niektorom z podpriečinkov.

V prípade, že projekt závisí na externej knižnici a táto tiež používa systém Forge,
môžeme ju špecifikovať pomocou plne kvalifikovaného mena cieľa (FQTN\footnote{Fully-Qualified Target Name}).
Tvar FQTN popisujeme v časti~\ref{sec:buildfiles:fqtn}.

\subsection{Plne Kvalifikovan\'{e} Meno Cie\v{l}a - FQTN}
\label{sec:buildfiles:fqtn}

Každý cieľ má meno, ktoré musí byť unikátne v rámci BUILD súboru v ktorom je definovaný.
Avšak nemôžeme vyžadovať, aby mená boli unikátne medzi všetkými projektmi, ktoré na
sebe vzájomne závisia. Preto systém Forge obsahuje tzv. FQTN, plne kvalifikované meno
cieľa. V princípe je FQTN ekvivalentné absolútnej ceste v priečinkovej štruktúre v OS UNIX\@.
Na rozlíšenie od absolútnej cesty sme rozhodli, že FQTN začína dvomi znakmi lomítka (\texttt{//}),
za ktorými nasleduje postupnosť mien priečinkov v tomto virtuálnom priečinkovom strome, oddelených
znakmi lomítka. Príkladmi FQTN sú napríklad: \texttt{//auvm/auvm}, \texttt{//auvm/lib/io}, alebo \\
\texttt{//github/imterra/forge/client/client}.

FQTN znižujú, ale neeliminujú úplne kolíziu mien cieľov, preto sme v časti~\ref{sec:buildfiles:bestpractices}
uviedli rady určené najmä pre používateľov verejných úložísk zdrojových kódov.

V prípade použitia systému Forge v rámci spoločnosti je odporúčané použiť doménové meno firmy
ako prefix pre hierarchiu projektov. Príkladom je naša open-source organizácia Imterra, ktorej všetky
FQTN začínajú s \texttt{//imterra/}.

\subsubsection{FORGE\_ROOT, BUILD súbory a FQTN}

V časti~\ref{sec:usage:client} popisujeme premennú prostredia \texttt{FORGE\_ROOT}, ktorá
špecifikuje koreňový priečinok systému Forge na danom počítači. FQTN špecifikuje výskyt
cieľa vzhľadom ku tomuto priečinku.

Meno cieľa je časť FQTN za posledným lomítkom. Pre nájdenie definície cieľa sa cieľ
s takýmto názvom hľadá v BUILD súbore, ktorý sa nachádza v podpriečinku \texttt{FORGE\_ROOT}
určeným cestou končiacou posledným lomítkom FQTN\@. Teda pre FQTN \texttt{//imterra/forge/test/foo/bar}
sa cieľ \texttt{bar} bude hľadať v súbore \\
\texttt{\$FORGE\_ROOT/imterra/forge/test/foo/build.yaml}.

\subsection{Odpor\'{u}\v{c}ania pre p\'{i}sanie BUILD s\'{u}borov}
\label{sec:buildfiles:bestpractices}

Pri písaní BUILD súborov odporúčame používať zjednodušený formát YAML a pre dobrú
čitateľnosť nevyužívať spätnú kompatibilitu s jazykom JSON. Pokiaľ by ale súbor bol
generovaný automatickým prevodom z iného formátu, je JSON formát akceptovateľný.

Pri zverejňovaní projektov používajúcich Forge na verejných úložískách zdrojových
kódov typu GitHub oporúčame používať FQTN začínajúce reťazcom \texttt{//github/USERNAME/REPOSITORY}\footnote{Adresa projektu na stránke bude \url{https://github.com/USERNAME/REPOSITORY}},
kde \texttt{USERNAME} je používateľské meno (alebo názov organizácie), ktorá repozitár
zverejnila a \texttt{REPOSITORY} je názov repozitára. Ekvivalentné pomenovanie odporúčame
aj pre ďalšie hostingové služby. Toto umožňuje používateľom systému závislého na
takto publikovanom systéme jednoducho stiahnuť zdrojové kódy bez potreby hľadania
na internete.

Pri formátovaní YAML súborov odporúčame využívať inline verziu zápisu poľa pre
polia s malým počtom hodnôt, ale pre dlhé hodnoty, alebo veľký počet hodnôt odporúčame,
pre zlepšenie čitateľnosti, definíciu poľa cez odrážky.

\section{Pou\v{z}itie syst\'{e}mu na kompil\'{a}ciu}
\label{sec:usage}

V tejto časti popisujeme spúšťanie a používanie systému Forge. Ako prvý uvádzame
Slave komponent, kedže bežiaci Slave komponent je potrebný pre správnu funkcionalitu. 

\subsection{Slave}
\label{sec:usage:server}

Slave komponent spúšťame, obvykle na serveri, pomocou príkazu \texttt{forge-server}.
Nastavenia koreňového priečinku, portu na ktorom je dostupný a počtu kompilačných
vlákien vykonávame podľa na to určených argumentov, ako sme uviedli v časti~\ref{sec:configuration:slave}.

Slave komponent sa na rozdiel od tradičných démonov nespúšťa na pozadí, preto je
odporúčané spustiť ho ako proces na pozadí pomocou operátora shellu \texttt{\&},
ako sme ukázali v listingu~\ref{lst:example:slave}, prípadne
pomocou zastavenia bežiaceho procesu a použitia príkazu \texttt{bg}.

\begin{listing}
  \begin{minted}[frame=lines,framesep=2mm,fontsize=\scriptsize]{sh}
    ~> forge-server --jobs=4 --port=4200 --root=/tmp/forgebuild &
  \end{minted}
  \caption[Ukážka spustenia Slave komponentu]{Ukážka spustenia Slave komponentu na pozadí s neštandardnými nastaveniami}
  \label{lst:example:slave}
\end{listing}

\subsection{Master}
\label{sec:usage:client}

Pri spustení Master komponentu mu, okrem prípadných nastavení pomocou argumentov,
ako uvádzame v časti~\ref{sec:configuration:slave}, zadávame aj názov cieľa alebo
niekoľko cieľov (oddelených medzerami), ktoré vyžadujeme splniť. Ciele uvádzame
pomocou ich FQTN, avšak, ak aktuálna pracovná zložka je podpriečinkom koreňového
adresára systému, môžeme uviesť aj relatívnu cestu k cieľu, od pracovnej zložky.
Obidva spôsoby adresácie sme ukázali v listingu~\ref{lst:example:master}.

V prípade spustenia Master komponentu s nastaveným parametrom \texttt{jobs} na hodnotu
inú ako 0, tento spustí ako podproces lokálny Slave komponent počúvajúci na porte 1103,
ktorý je po ukončení vykonávania Master komponentu taktiež ukončený.

\begin{listing}
  \begin{minted}[frame=lines,framesep=2mm,fontsize=\scriptsize]{sh}
    /home/user/src/foo> forge --root=/home/user/src --worker 10.0.1.7:1103 //auvm/auvm foo
  \end{minted}
  \caption[Ukážka kompilácie pomocou Forge]{Kompilácia cieľa auvm pomocou FQTN a lokálneho cieľa foo, s použitím vzdialeného Slave komponentu}
  \label{lst:example:master}
\end{listing}

\section{Roz\v{s}irovanie funkcionality syst\'{e}mu}
\label{sec:contributing}

Dôležitou schopnosťou systému Forge musí byť rozšíriteľnosť. Preto v tejto časti popisujeme
spôsoby na implementáciu 2 najčastejších rozšírení: akcií a typov cieľov.

\subsection{Prid\'{a}vanie nov\'{y}ch akci\'{i}}
\label{sec:contributing:actions}

Implementácia novej akcie je prevedená najmä na strane Slave komponentu. Na jej sprístupnenie
používateľom ju treba implementovať ako metódu typu \texttt{Task} z modulu \texttt{tasks}.
Táto metóda musí spĺňať všetky kritériá pre metódu RPC služby. V prípade, že by
typ \texttt{Task} nespĺňal všetky požiadavky pre akciu, napríklad neobsahuje isté
informácie, je možné metódu implementovať na inom, novom, type, ktorý je ale potom
potrebné zaregistrovať ako RPC službu v hlavnom module Slave komponentu.

Na strane Master komponentu nie je akciu potrebné implementovať, je potrebné len
uviesť správnu RPC metódu pri jej vytvárani z cieľa. Ako vytvoriť nový typ cieľa uvádzame
v ďalšej časti.

\subsection{Prid\'{a}vanie nov\'{y}ch typov cie\v{l}ov}
\label{sec:contributing:targets}

Pridávanie nových typov cieľov prebieha výlučne na strane Master komponentu. Na úspešnú
implementáciu je potrebné pridať jeho podporu do spracovávnia BUILD súborov a následne
definovať vytvorený objekt, tak aby tento spĺňal požiadavky rozhrania \texttt{Target}.

Rozhranie \texttt{Target} vyžaduje implementáciu 5 metód, kde najdôležitejšia pre
splnenie cieľa je metóda \texttt{GetOutputFile}, ktorá prekladá cieľ do grafu súborov
a akcií na nich prevedených. Odporúčame nový cieľ implementovať pomocou typu
definovaného v module \texttt{target}, kedže tento modul interne sprístupňuje
hešovaciu tabuľku všetkých doteraz vytvorených súborov. Táto sa používa na zabránenie
vytvorenia viacerých objektov typu \texttt{File}, ktoré reprezentujú rovnaký fyzický
súbor a teda následné problémy s opakovanou kompiláciou. Táto tabuľka je dostupná
ako modulu prístupná premenná \texttt{file\_list}. Jej kľúčmi sú cesty k súborom
relatívne ku koreňovému adresáru systému Forge.

Na sprístupnenie takto implementovaného typu cieľa tvorcom BUILD súboru, je potrebné
o ňom informovať funkciu \texttt{ParseFile}, ktorá spracúva tieto súbory. Na konci
funkcie sa nachádza \texttt{switch-case} výraz, ktorý na základe textovej reprezentácie
typu cieľa v BUILD súbore vráti objekt potrebného typu. Na spracovanie YAML údajov,
ktoré sú dostupné sme pre typy \texttt{lib\_c} a \texttt{app\_c} implementovali
pomocné funkcie \texttt{MakeLibCTarget} a \texttt{MakeAppCTarget}. Odporúčame inšpirovať
sa ich implementáciou pri tvorbe nových pomocných funkcií.

Pre zjednodušenie vytvárania závislostí v podobe ďalších cieľov sme vytvorili
funkciu \texttt{MakeDependencies}, ktorá z YAML údajov spracuje pole \texttt{dependencies}
a následne opakovaným volaním funkcie \texttt{ParseFile}, vytvorí tieto ciele. Použitie
tejto funkcie je silne odporúčané.
