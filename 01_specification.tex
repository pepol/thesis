\chapter{\v{S}pecifik\'{a}cia a n\'{a}vrh syst\'{e}mu}
\label{ch:spec}

V tejto kapitole popíšeme požiadavky na systém a navrhneme riešenie.

\section{\'{U}vod do problematiky}
\label{sec:intro}

Majme projekt skladajúci sa z väčšieho množstva zdrojových súborov, ktoré treba
skompilovať (alebo pre webové aplikácie minifikovať) predtým, ako z nich vytvoríme
výslednú aplikáciu. Keďže postup kompilácie je presne daný a častokrát opakovaný,
je veľmi výhodné ušetriť čas potrebný na spúšťanie všetkých príkazov za sebou
použitím nejakého systému, ktorý to zautomatizuje. Okrem automatizácie chceme
určite ušetriť aj čas pri takzvanej rekompilácii, aby sa po zmene niekoľkých súborov
nemusela znova kompilovať celá aplikácia. Niekoľko existujúcich systémov spomíname
v časti~\ref{sec:existing}.

V prípade spoločnosti pracujúcej na projektoch obsahujúcich rádovo stovky až tisíce
súborov je takýto systém nevyhnutný. Avšak aj pri použití automatizácie a použitia
výsledkov predchádzajúcich behov môže rekompilácia, najmä pri väčších zmenách,
trvať dlho, čo znižuje nielen produktivitu zamestnancov, ale aj zisky spoločnosti.
Preto niektoré korporácie využívajú na kompiláciu (ktorá je pri veľkom počte
nezávislých súborov inherentne vysoko paralelizovateľná) nielen pracovné stanice
zamestnancov, ale aj ich serverové farmy.

Avšak narážame na problém. V čase písania tejto práce sú všetky takto distribuované
riešenia uzavreté, neprístupné verejnosti, iba spoločnosti, ktorá ho vyvinula
a drží si ho pod svojou strechou. Toto je veľká nevýhoda pre začínajúce start-up
organizácie, ktoré, aj keď majú k dispozícii servery, ktoré by takto mohli využiť,
nemôžu si dovoliť vývoj takéhoto riešenia pre vlastnú potrebu.

\section{Formul\'{a}cia probl\'{e}mu}
\label{sec:problem}

Sformulujme teda problém na základe predchádzajúcej časti a pridajme ďalšie
požiadavky:

\begin{itemize}
  \item Automatizácia kompilácie a linkovania zdrojových súborov do výslednej aplikácie.
  \item Využívanie predchádzajúcich výsledkov pri ďalších behoch.
  \item Distribúcia kompilácie na servery, pokiaľ sú dostupné.
  \item Sprístupnenie existujúcich behov a výsledkov celej organizácii.
  \item Rozšíriteľná podpora rôznych jazykov a postupov.
  \item Minimalizovanie potreby ručnej konfigurácie.
  \item Podpora VCS\footnote{Version Control System}, najmä Git.\footnote{\url{https://git-scm.com}}
  \item Vydané pod open-source licenciou.
\end{itemize}

Pozrime sa podrobnejšie na niektoré požiadavky.

\subsubsection{Sprístupnenie existujúcich behov a výsledkov celej organizácii}
Ak dvaja zamestnanci pracujú na projekte potrebujúcom kompiláciu tisícky súborov,
nie je dôvod aby obaja, pri prvom pokuse skompilovať takýto projekt, museli čakať
niekoľko hodín pred tým ako môžu ďalej pracovať. Preto plánujeme vytvoriť distribuované
úložisko objektov vytvorených kompiláciou na serveroch, ktoré budú takto dostupné
všetkým používateľom daného systému. Toto všetko za účelom čo najväčšieho ušetrenia
výpočtového výkonu a času spotrebovaného kompiláciou.

\subsubsection{Rozšíriteľná podpora rôznych jazykov a postupov}
V prvej verzii tohto systému plánujeme implementovať podporu pre niekoľko najpoužívanejších
jazykov. Avšak uvedomujeme si, že takýto systém by mal byť nezávislý na použitých
nástrojoch a preto jeho súčasťou bude aj jednoduchý spôsob pridávania podpory
pre iné platformy.

\subsubsection{Minimalizovanie potreby ručnej konfigurácie}
V prípade existujúcich riešení je častokrát až zarážajúce množstvo konfigurácie,
ktorú treba vykonať pri použití takého systému v projekte. Náš systém by mal
tieto záležitosti čo najviac zjednodušiť.

\subsubsection{Podpora VCS}
Veľké množstvo projektov závisí na súboroch, knižniciach, ktoré sa nenachádzajú
v ich repozitári, ale sú dostupné pod nejakým VCS\@. Náš systém by mal, pri
vhodnej špecifikácii dostupnosti týchto závislostí v konfigurácii, byť schopný
automaticky stiahnuť a použiť pri kompilácii takúto závislosť, bez potreby akejkoľvek
činnosti od používateľa.

\section{Existuj\'{u}ce syst\'{e}my}
\label{sec:existing}
V tejto časti popíšeme niekoľko open-source build systémov. Vo všetkých prípadoch
ide ale iba o lokálnu kompiláciu.

\subsubsection{GNU Make}
GNU Make\footnote{\url{https://www.gnu.org/software/make/}} je najstarší a stále
pravdepodobne najpoužívanejší build systém. Využíva súbor nazvaný Makefile, ktorý
obsahuje popis pravidiel na vytváranie súborov na základe ich prípony. Na zabránenie
opakovanej kompilácii porovnáva časy posledného zápisu do súborov (zdrojového a
výsledného), čo ale nezabraňuje opakovanej kompilácii napríklad v prípade obnovenia
zdrojového súboru zo zálohy. Tento systém podporuje paralelnú kompiláciu, ale len
v obmedzenej miere a na lokálnom počítači. Na zjednodušenie písania pravidiel v Makefile
vzniklo niekoľko systémov, ktoré tvorbu Makefile automatizujú. Medzi najpoužívanejšie
patria GNU Autotools a CMake.

\subsubsection{Apache Ant}
Apache Ant\footnote{\url{https://ant.apache.org/}} je build systém, ktorý adresuje
niekoľko problémov s prenositeľnosťou Makefile súborov medzi platformami, keďže
tieto špecifikujú akcie ako postupnosť príkazov v shelli. Konfigurácia je vykonávaná
v XML súboroch, ktoré definujú strom úloh (tasks) pre projekt. Každá z úloh je
implementovaná ako trieda v jazyku Java, v ktorom je implementovaný aj celý Ant
a teda je jednoduché pridať nové úlohy, ako aj preniesť existujúce úlohy na nové
systémy. Tento systém využívajú najmä projekty písane v jazyku Java.

\subsubsection{Bazel}
Open-source verzia interného build systému spoločnosti Google, ktorý bol prvý krát
zverejnený v roku 2014. Využíva podobné koncepty ako Apache Ant. Funkcionalita tohto
systému bola popísaná v roku 2012 sériou článkov na Google Developers blogu
\cite{BlazeDesign} a slúži ako inšpirácia pri návrhu nášho systému. Podľa plánov
vydania \cite{BazelRoadmap} sa ale Google nechystá zverejniť ich implementáciu
distribuovanej kompilácie, ktorá je popísaná vo vyššie spomínanej sérii článkov.

\section{N\'{a}vrh rie\v{s}enia}
\label{sec:solution}
V tejto časti popíšeme jednotlivé komponenty systému a ich funkcionalitu.

\subsection{Ciele kompil\'{a}cie}
\label{sub:solution:targets}

Kompilovaný projekt delíme na ciele kompilácie (anglicky targets), ktoré definujú
jednotlivé základné stavebné prvky objektu. Každý cieľ má určený svoj typ, napríklad
aplikácia písaná v jazyku C, alebo knižnica funkcií jazyka C++. Okrem typu môžeme
cieľu určiť závislosť na splnení iných cieľov, alebo špecifikovať vstupné zdrojové
súbory a súbory prostriedkov.

Výsledkom splnenia každého cieľa kompilácie je generovaný súbor. Tento súbor vzniká
aplikáciou pravidiel (akcií) na zdrojové súbory cieľa alebo iných generovaných súborov.
Jeden cieľ môže špecifikovať aj postupnosť niekoľkých pravidiel.

Z grafu závislostí medzi cieľmi kompilácie sa vytvorí bipartitný graf pravidiel a súborov,
ktorý popisujeme nižšie.

\subsubsection{Zdrojové súbory}
Toto sú súbory, ktoré sa nachádzajú priamo v repozitári projektu, sú vytvorené
ľuďmi (alebo nástrojmi) a nie generované týmto systémom. V grafe závislostí sú
koncovými vrcholmi, keďže nemajú žiadne závislosti.

\subsubsection{Generované súbory}
Súbory vytvorené ako výstup z nejakého pravidla, častokrát slúžia zároveň aj ako
vstupné súbory pre ďalšie akcie.

\subsubsection{Pravidlá}
Popisujú vzťah a postup vytvorenia výstupného súboru zo súborov vstupných.
Každé pravidlo má práve jeden výstupný súbor a 0 alebo viac vstupných. Pravidlá
zahŕňajú postupy ako kompilácia, linkovanie a podobne.

\subsubsection{Graf závislostí}

Pri vyžiadaní kompilácie cieľa vytvoríme orientovaný graf závislostí, ktorý vyžadujeme,
aby bol acyklický. V prípade nájdenia cyklu systém ukončí beh s chybou o nesplniteľnosti
požiadavky.

Vrcholy grafu tvoria súbory a pravidlá, kde hrana od súboru k pravidlu existuje
práve vtedy, keď je súbor vstupom pravidla, a od pravidla k súboru, ak je súbor
výstupom pravidla. Takto vytvorený graf bude bipartitný, kde disjunktnými množinami
sú pravidlá a súbory. Na obrázku~\ref{img:dependency_graph} sa nachádza príklad
takéhoto grafu.

\begin{figure}[h]
  \centerline{\includegraphics[width=0.6\textwidth]{images/dependency_graph}}
  \caption[Graf závislostí]{Príklad grafu závislostí, okrúhle vrcholy predstavujú
  súbory, štvorcové vrcholy predstavujú pravidlá.}
  \label{img:dependency_graph}
\end{figure}

Takto zostrojený graf bude potom použitý slave komponentmi na vytvorenie výstupných
súborov s použitím pravidiel a príkazov ktoré so sebou nesú.

\subsection{Master}
\label{sub:solution:master}
Aplikáciu bude predstavovať program, bežiaci na užívateľskom počítači. Tento program
nazvime {\it master}.

Úlohou mastera bude v okamihu zavolania a špecifikovania cieľa kompilácie spracovať
konfiguráciu všetkých cieľov daného projektu a vytvoriť strom závislostí. Následne master
kontaktuje jednotlivé {\it slave} \/komponenty bežiace na rôznych strojoch (lokálny
počítač, serverová farma a pod.) a rozdistribuuje ciele na tieto počítače čo možno
najefektívnejšie.

Pokiaľ cieľ alebo podciele závisia na vonkajšom projekte a repozitár tohto projektu
je dostupný, tento projekt je stiahnutý a použitý. V prípade nedostupnosti niektorej
zo závislostí program ukončí beh s chybovou hláškou.

\subsection{Slave}
\label{sub:solution:slave}
Slave je program starajúci sa o vykonávanie úloh zadaných masterom. Na fyzickom počítači,
či už lokálnom, alebo na serveri bude bežať zvyčajne niekoľko slave programov, nakoľko
jeden slave program bude navrhnutý na vykonávanie úloh pre práve jeden cieľ. Keď voľný
slave obdrží úlohu, v prvom kroku sa pozrie do zdieľaného úložiska objektov, ak je
toto dostupné, či už žiadaný súbor neexistuje. V prípade jeho existencie slave
pošle výsledný súbor priamo masterovi a nebude vykonávať kompiláciu. V prípade
neexistencie výsledného súboru v žiadanej verzii bude cieľ preložený na postupnosť
akcií a tieto budú vykonané.

\subsection{\'{U}lo\v{z}isko objektov}
\label{sub:solution:storage}
Úložisko objektov slúži ako cache na zabránenie opakovanej kompilácii rovnakých
súborov. Výsledné súbory tu budú nahrané po dokončení kompilácie slave programom.
Ukladanie bude prebiehať podľa názvu zdrojového súboru a jeho kontrolného súčtu,
aby sa ľahko rozlíšili rôzne verzie súboru, napríklad lokálne modifikovaná od
tzv. HEAD verzie\footnote{najnovšia verzia zverejnená v hlavnom repozitári}.

V prípade nášho úložiska sme sa rozhodli použiť na kontrolné súčty hashovaciu funkciu
SHA512, kvôli čo najväčšej minimalizácii kolízií.

\subsection{Dostupnosť zdrojových súborov}
\label{sub:solution:source}
Zdrojový kód je verzionovaný systémom Git\footnote{\url{https://git-scm.com}} a
zverejnený pod licenciou MIT v GitHub repozitári na adrese \url{https://github.com/imterra/forge}.
