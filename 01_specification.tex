\chapter{\v{S}pecifik\'{a}cia a n\'{a}vrh syst\'{e}mu}
\label{ch:spec}

V tejto kapitole vyslovíme požiadavky na systém a navrhneme riešenie.

\section{\'{U}vod do problematiky}

Majme projekt skladajúci sa z väčšieho množstva zdrojových súborov, ktoré treba
skompilovať (alebo pre webové aplikácie minifikovať) predtým, ako z nich vytvoríme
výslednú aplikáciu. Keďže tento postup je presne daný a častokrát opakovaný,
je veľmi výhodné ušetriť čas potrebný na písanie všetkých príkazov za sebou
použitím nejakého systému, ktorý to zautomatizuje. Okrem automatizácie chceme
určite ušetriť aj čas pri takzvanej rekompilácii, aby sa po zmene niekoľkých súborov
nemusela znova kompilovať celá aplikácia. Niekoľko existujúcich systémov spomíname
v~\ref{sec:existing}.

V prípade spoločnosti pracujúcej na projektoch obsahujúcich rádovo stovky až tisíce
súborov je takýto systém nevyhnutný. Avšak aj pri použití automatizácie a použitia
výsledkov predchádzajúcich behov môže rekompilácia, najmä pri väčších zmenách,
trvať dlho, čo znižuje nielen produktivitu zamestnancov, ale aj zisky spoločnosti.
Preto niektoré korporácie využívajú na kompiláciu (ktorá je pri veľkom počte
nezávislých súborov inherentne vysoko paralelizovateľná) nielen pracovné stanice
zamestnancov, ale, v prípade internetových gigantov aj ich serverové farmy.

Avšak narážame na problém. V čase písania tejto práce sú všetky takto distribuované
riešenia uzavreté, neprístupné verejnosti, iba spoločnosti, ktorá ho vyvinula
a drží si ho pod svojou strechou. Toto je veľká nevýhoda pre začínajúce start-up
organizácie, ktoré, aj keď majú k dispozícii servery, ktoré by takto mohli využiť,
nnemôžu si dovoliť vývoj takéhoto riešenia.
