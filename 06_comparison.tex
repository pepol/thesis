\chapter{Testovanie syst\'{e}mu}
\label{ch:comp}

V tejto kapitole otestujeme a porovnáme výkon systému na zdrojových kódoch
existujúcich open-source projektov. Porovnáme čas behu kompilácie pomocou štandardného
kompilačného programu GNU Make, pomocou Forge kompilujúceho iba lokálne a pomocou
Forge kompilujúceho pomocou 2 Slave komponentov, jedného bežiaceho na lokálnom počítači
a druhého bežiaceho na vzdialenom serveri.

\section{V\'{y}ber projektov na testovanie}
\label{sec:projectchoice}

Kedže Forge podporuje zatiaľ iba kompiláciu zdrojových súborov jazyka C, rozhodli
sme sa použiť projekt písaný iba v jazyku C. Druhým kritériom výberu bola jednoduchosť
organizácie projektu a automatizovateľnosť tvorby BUILD súboru. Zároveň sme ale
vyžadovali, aby bol projekt rozsiahly, vzhľadom na povahu testovania. Pri
malom, napríklad jednosúborovom projekte by nebola distribúcia kompilácie použiteľná,
prípadne by väčšiu rolu vo výsledkoch hralo aktuálne zaťaženie počítača. Toto by
mohlo mať za následok skreslenie skutočných výsledkov.

Pre testovanie sme vybrali projekt redis\footnote{\url{https://redis.io}}, ktorý
spĺňal všetky naše požiadavky. Tento projekt na kompiláciu používa iba štandardný
nástroj GNU Make, bez použitia GNU Autotools, čo značne zjednodušilo tvorbu BUILD
súboru.

\section{Postup testovania}
\label{sec:process}

Pre každú z testovaných konfigurácií sme spustili kompiláciu programu \texttt{redis-server}.
Kompiláciu sme opakovali pre každý program 5 krát, aby sme zabránili skresleniu výsledkov
kvôli vyťaženiu počítača. Všetky opakované kompilácie boli spúšťané na čistých zdrojových
kódoch, aby sme zabezpečili reálne spustenie kompilácie.

\subsection{Parametre testovania}
\label{sec:process:params}

Testovanie prebiehalo na laptope s operačnou pamäťou s kapacitou 16GB a dvojjadrovým procesorom Intel Core i7-5500U s
frekvenciou 2.4GHz a podporou technológie HyperThreading, ktorá spôsobuje, že procesor je schopný pracovať
ako štvorjadrový. Pri použití programu GNU Make, ako aj lokálneho Slave komponentu
systému Forge sme preto nastavili počet pracujúcich vlákien na 4.

Vzdialený Slave komponent bežal na serveri so štvorjadrovým procesorom Intel Atom C2750 s
frekvenciou 2.4GHz a operačnou pamäťou s kapacitou 4GB.

\subsection{Vytvorenie BUILD s\'{u}boru}
\label{sec:process:buildfile}

BUILD súbor pre projekt redis sme vytvorili na základe originálneho súboru Makefile,
ktorý sa štandardne používa pri kompilácii pomocou nástroja GNU Make. Tento súbor
obsahuje pravidlo, ktoré zabezpečí generáciu tzv.\ súboru závislostí, popisujúceho
všetky súbory, ktoré zdrojový súbor jazyka C používa pomocou direktívy \texttt{\#include}.

Takto vytvorený súbor závislostí sme potom skriptom \texttt{dep2buildfile.sh}\footnote{Tento skript je zverejnený spolu so zdrojovými kódmi programu Forge}
skonvertovali na takmer hotový BUILD file. Tento skript ale vygeneruje len ciele pre jednotlivé knižnice
programu. Následne sme pridali ďalší cieľ reprezentujúci výslednú aplikáciu \texttt{redis-server},
ktorá závisela na knižniciach, podľa špecifikácie cieľa \texttt{redis-server} v súbore Makefile.

\subsection{Meranie \v{c}asu}
\label{sec:process:time}

Meranie času sme zabezpečili pomocou štandardnej utility OS Linux time\footnote{\url{http://linux.die.net/man/1/time}}.
Táto utilita na príkazovom riadku očakáva príkaz, ktorého čas vykonávania meria. Okrem celkového
času vykonávania zbiera aj hodnoty o vyťažení CPU (toto vyťaženie môže dosiahnuť
maximálne hodnotu $N\times100\%$, kde N je počet logických CPU na počítači.

\section{V\'{y}sledky testovania}
\label{sec:results}

V tejto časti uvádzame prehľad nameraných výsledkov, spolu s vysvetleniami a poznámkami.
V tabuľkách sme uviedli prehľad vyťaženia procesorov a dĺžku trvania behu programu pre používateľa,
to znamená od spustenia programu po jeho úspešné ukončenie. Posledný riadok tabuľky obsahuje
priemerné hodnoty.

\subsubsection{GNU Make}
\label{sec:results:make}

\begin{table}[H]
  \centering
  \begin{tabular}{| l || c | r |}
    \hline
    \v{C}\'{i}slo pokusu & Vyu\v{z}itie CPU & \v{C}as \\ \hline \hline
    1. & 332\% & 21.74s \\ \hline
    2. & 318\% & 21.79s \\ \hline
    3. & 324\% & 21.72s \\ \hline
    4. & 272\% & 21.78s \\ \hline
    5. & 289\% & 21.69s \\ \hline \hline
    Priemer & 307\% & 21.74s \\
    \hline
  \end{tabular}
  \caption{V\'{y}sledky pre GNU Make}
  \label{tbl:make}
\end{table}

\subsubsection{Nedistribuovan\'{y} Forge}
\label{sec:results:forgelocal}

\begin{table}[H]
  \centering
  \begin{tabular}{| l || c | r |}
    \hline
    \v{C}\'{i}slo pokusu & Vyu\v{z}itie CPU & \v{C}as \\ \hline \hline
    1. & 270\% & 6.85s \\ \hline
    2. & 323\% & 6.90s \\ \hline
    3. & 317\% & 6.71s \\ \hline
    4. & 335\% & 6.96s \\ \hline
    5. & 332\% & 6.87s \\ \hline \hline
    Priemer & 315\% & 6.85s \\
    \hline
  \end{tabular}
  \caption{V\'{y}sledky pre nedistribuovan\'{y} Forge}
  \label{tbl:forgelocal}
\end{table}

Z výsledkov vidíme, že systém Forge je asi 3-krát rýchlejší ako GNU Make. Toto je
pravdepodobne kvôli tomu, že Make spúšťa príkazy kompilácie nepriamo, spustením shellu,
ktorý potom spúšťa samotný kompilátor, na rozdiel od systému Forge, ktorý spúšťa kompilátor
priamo ako dcérsky proces. Ďalším dôvodom môže byť samotné parsovanie build programov, kde
Makefile je oveľa zložitejšie spracovávať, najmä kvôli rôznym definíciam, premenným a volaniam
externých príkazov. Formát YAML, ktorý používa pre svoje BUILD súbory Forge je jednoducho
a rýchlo spracovateľný.

\subsubsection{Distribuovan\'{y} Forge}
\label{sec:results:forge}

Pri testovaní použitia serveru na kompiláciu sme sa rozhodli vyskúšať 2 varianty nastavení.
Obe nastavenia využívajú plnú kapacitu serveru, pričom prvé zároveň využíva aj plnú kapacitu
testovacieho laptopu, zatiaľčo druhé využíva laptop iba z polovice.

\begin{table}[H]
  \centering
  \begin{tabular}{| l || c | r |}
    \hline
    \v{C}\'{i}slo pokusu & Vyu\v{z}itie CPU & \v{C}as \\ \hline \hline
    1. & 263\% & 6.22s \\ \hline
    2. & 302\% & 6.04s \\ \hline
    3. & 278\% & 5.68s \\ \hline
    4. & 300\% & 6.20s \\ \hline
    5. & 302\% & 6.25s \\ \hline \hline
    Priemer & 289\% & 6.08s \\
    \hline
  \end{tabular}
  \caption{V\'{y}sledky pre distribuovan\'{y} Forge so 4 lok\'{a}lnymi vl\'{a}knami}
  \label{tbl:forge4}
\end{table}

V porovnaní s predchádzajúcim, čisto lokálnym behom systému vidíme mierny pokles
trvania kompilácie, ako aj pokles vyťaženia procesora, presne ako bolo očakávané.
Zrýchlenie spôsobené zdvojnásobení počtu kompilujúcich vlákien by sa prejavilo
viditeľnejšie pri väčšom projekte.

\begin{table}[H]
  \centering
  \begin{tabular}{| l || c | r |}
    \hline
    \v{C}\'{i}slo pokusu & Vyu\v{z}itie CPU & \v{C}as \\ \hline \hline
    1. & 25\% & 2.45s \\ \hline
    2. & 18\% & 2.69s \\ \hline
    3. & 28\% & 2.46s \\ \hline
    4. & 26\% & 2.41s \\ \hline
    5. & 26\% & 2.42s \\ \hline \hline
    Priemer & 24\% & 2.48s \\
    \hline
  \end{tabular}
  \caption{V\'{y}sledky pre distribuovan\'{y} Forge s 2 lok\'{a}lnymi vl\'{a}knami}
  \label{tbl:forge2}
\end{table}

Výsledok behu s 2 lokálnymi vláknami môže vyzerať veľmi prekvapivo. Hlavným dôvodom
tohto zrýchlenia je pravdepodobne uvoľnenie jedného z lokálnych vlakien na spracovávanie
stálej sieťovej komunikácie so serverom a teda využívanie reálnych 6 jadier. Pri použití
4 lokálnych vlákien je totiž vlákno zabezpečujúce komunikáciu blokované vláknom so spusteným
kompilačným procesom, čo môže spôsobovať spomalenie systému.

Pre výsledky lepšie vypovedajúce o efektívnosti implementácie by bol potrebný test
kompilácie projektu s rádovo stovkami súborov na desiatkach serverov. Problémom takéhoto
testu je ale náročnosť na prostriedky. Druhým problémom s ktorým sme sa stretli bolo
nájdenie vhodného, dostatočne veľkého projektu. Z týchto dôvodov sme tento test neuskutočnili.
