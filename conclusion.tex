\chapter*{Záver}
\addcontentsline{toc}{chapter}{Záver}

Úspešne sa nám podarilo implementovať jednoduchý distribuovaný build systém napísaný
v jazyku Go. Pri implementácii systému sme sa držali konvencií jazyka Go a snažili
sme sa všetky zdroje používateľského počítača využívať čo možno najefektívnejšie.
Systém Forge sme zverejnili ako open-source pod MIT Licenciou na stránke
\url{https://github.com/imterra/forge}.

Z výsledkov testov v kapitole~\ref{ch:comp} vidíme, že systém je naprogramovaný
efektívne a dokázal poskytnúť takmer 9-násobné zrýchlenie iba s použitím jedného
servera oproti lokálne bežiacemu systému GNU Make.

Najväčším problémom s ktorým sme sa stretli bol návrh algoritmu na distribúciu
kompilačných úloh na servery. Pri našej implementácii sme použili jednoduchý
naivný algoritmus na základe vyťaženosti stroja a dostupnosti súborov. Pre
efektívnejšiu distribúciu by bolo možné použiť napríklad algoritmus založený
na poznatkoch zo strojového učenia.

V budúcnosti by sme radi rozšírili systém o podporu ďalších programovacích jazykov,
ako aj zlepšili podporu aktuálne podporovaných. Taktiež by sme radi pridali
podporu na zabezpečenú komunikáciu medzi komponentmi, zdieľané úložisko súborov
a podporu pre systémy správy verzií zdrojových kódov, aby bola aplikácia reálne
použiteľná v komerčnej sfére.
