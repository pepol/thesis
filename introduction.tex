\chapter*{Úvod}
\addcontentsline{toc}{chapter}{Úvod}

Kompilácia zdrojových súborov je každodenná súčasť práce na softvérovom projekte.
Opakované zadávanie rovnakých príkazov môže pri nepozornosti spôsobenej rutinou
spôsobiť zlyhanie. Preto je potrebné takéto procesy automatizovať. V spoločnostiach
s prístupom k serverovej farme je možné tento výkon, pokiaľ ho nevyužívajú používatelia
použiť na urýchlenie tohto procesu.

Existujúce riešenia ako GNU Make, Apache Ant alebo Bazel poskytujú automatizáciu
procesu na počítači používateľa. Avšak pre spoločnosti s prístupom k serverovým
farmám a s rozsiahlymi projektami je výhodné použiť nevyužitý výkon serveru
na kompiláciu. Niektoré veľké firmy majú vlastné systémy ktoré toto umožňujú,
ale neexistuje jednoduché open-source riešenie, ktoré by vznikajúce spoločnosti
mohli využiť.

Preto sme sa rozhodli implementovať distribuovaný build systém, ktorý bude jednoduchý
na použitie a výkonný aj pri rozsiahlych softvérových projektoch. Ako vhodný programovací
jazyk sa javí Go, vďaka vstavanej podpore paralelizmu, a ako platformu sme zvolili
operačný systém Linux/Unix, kvôli rozšírenosti na serveroch ako aj pracovných staniciach
programátorov.

Cieľom tejto práce je poskytnúť teoretické základy za systémom a tiež oboznámiť
čitateľa so štruktúrou a logikou aplikácie.

V kapitole~\ref{ch:spec} popíšeme požiadavky a návrh systému. V kapitole~\ref{ch:tech}
vysvetlíme dôvody výberu použitých technológií a v kapitole~\ref{ch:theory} uvedieme
teoretické základy návrhu.

Kapitola~\ref{ch:impl} tvorí jadro práce a popisujeme v nej implementáciu aplikácie.
Vysvetlíme fungovanie a logiku systému, popíšeme jednotlivé moduly a služby, a uvedieme
ukážky zaujímavých časti zdrojového kódu.

V kapitole~\ref{ch:doc} poskytneme dokumentáciu inštalácie, nastavenia a používania systému, spoločne
s príkladmi popisov kompilovanych projektov. Na záver, v kapitole~\ref{ch:comp} porovnáme
vytvorený systém s existujúcimi riešeniami.
