\chapter{Pou\v{z}it\'{e} technol\'{o}gie}
\label{ch:tech}

V tejto časti sa bližšie pozrieme na použité technológie, dôvody na ich využitie,
ich výhody a nevýhody.

\section{Go}
\label{sec:golang}

Jazyk Go je kompilovaný, staticky typovaný programovaný jazyk pôvodne vyvinutý
spoločnosťou Google. Jeho syntax vychádza zo syntaxe jazyka C. Jazyk poskytuje
automatickú správu pamäte (garbage collection), limitované prvky objektovo orientovaného
programovania a podporuje súbežnosť na báze komunikujúcich súbežných procesov.\cite{GolangFeatures}

\subsection{Objektovo orientované programovanie}
\label{sec:golang:oop}

Jazyk Go nepodporuje existenciu objektov v štýle OOP jazykov ako Java. Namiesto toho
sa OOP v jazyku Go implementuje pomocou štruktúr a metód s tzv.\ typovým prijímačom
(method receiver). Príklad uvádzame v listingu~\ref{lst:golang:receivers}. Funkcia
Area je metódou typu Square.

\begin{listing}[h]
  \begin{minted}[frame=lines,framesep=2mm,linenos,fontsize=\scriptsize]{go}
    type Square struct {
            side float64
    }

    func (sq Square) Area() float64 {
            return sq.side * sq.side
    }
  \end{minted}
  \caption{Ukážka implementácie metódy v jazyku Go.}
  \label{lst:golang:receivers}
\end{listing}

Zapuzdrenie sa v Go rieši pomocou exportovaných a neexportovaných prvkov štruktúry.
Táto konvencia sa netýka iba štruktúr, ale aj funkcií exportovaných v rámci modulu.
Názvy exportovaných metód a prvkov začínajú veľkým písmenom. Pri použití neexportovanej
časti modulu, alebo štruktúry skončí kompilácia programu chybou.

\begin{listing}[h]
  \begin{minted}[frame=lines,framesep=2mm,linenos,fontsize=\scriptsize]{go}
    import "math"

    type Shape interface {
            Area() float64
    }

    type Square struct {
            side float64
    }

    func (sq Square) Area() float64 {
            return sq.side * sq.side
    }

    type Circle struct {
            radius float64
    }

    func (c Circle) Area() float64 {
            return math.Pi * math.Pow(c.radius, 2)
    }
  \end{minted}
  \caption{Použitie rozhrania na implementáciu dedičnosti v jazyku Go.}
  \label{lst:golang:inheritance}
\end{listing}

Dedičnosť sa v Go rieši dvomi spôsobmi. Prvým z nich je použitie skladania, kedy sa
do podtriedy vloží anonymný prvok nadtriedy, z ktorej dedíme. Toto umožňuje kompozíciu
a delegáciu objektov. Druhým spôsobom je definícia rozhrania (interface). Rozhranie je
popísanie skupiny funkcií. Akýkoľvek typ, ktorý implementuje všetky funkcie rozhrania
je považovaný za objekt tohto rozhrania. Kontrola, či objekt spĺňa
požiadavky rozhrania nastáva v jazyku Go počas kompilácie, vývojári jazyka túto
metódu nazývajú štruktúralne typovanie a vychádza z protokolov jazyka Smalltalk.\cite{GolangInterface}
Príklad uvádzame v listingu~\ref{lst:golang:inheritance}.
Rozhranie Shape vyžaduje implementáciu funkcie Area, ktorú implementuje aj typ Square,
aj typ Circle a teda obe môžeme použiť ako objekt typu Shape. 

\subsection{Súbežnosť v jazyku Go - Goroutines, channels}
\label{sec:golang:concurrency}

Jednou z veľkých výhod jazyka Go oproti iným jazykom určeným na podobné použitie,
ako je C++ je vstavaná podpora pre súbežnosť, ktorá sa dá dosiahnuť použitím
dvoch vlastností jazyka, tzv.\ goroutines a údajového typu kanál (channel).

Goroutine je funkcia, ktorá sa volá pomocou kľúčového slova \texttt{go}. Táto
funkcia je spustená v samostatnom ľahkom vlákne. Tieto vlákna ale nemôžeme považovať
za ekvivalent vlákien - threads, v OS, kedže sú oveľa lacnejšie na vytvorenie a použitie.
V implementácii sú takto bežiace goroutines multiplexované na reálne thready OS bežiacej
aplikácie.

Na synchronizáciu medzi goroutines sa používa údajový typ kanál (channel). Jedná sa o
implementáciu dátovej štruktúry rad (queue) so správnou synchronizáciou prístupu,
aby sa zabránilo kritickým súbehom. Súčasťou definície každého kanálu je aj údajový
typ, ktorého objekty sa pomocou neho posielajú. Kanál má určenú kapacitu, prednastavená
je hodnota 1. V prípade pokusu o zápis do plného kanálu táto operácia blokuje, tak isto
ako pri pokuse o čítaní z prázdneho kanálu. Vzhľadom na to je jednoduché implementovať
napríklad synchronizáciu pomocou semafórov.

\subsection{RPC}
\label{sec:golang:rpc}

Súčasťou štandardnej knižnice jazyka Go je aj modul \texttt{net/rpc}, ktorý implementuje
volanie vzdialených procedúr (remote procedure call - RPC). Táto služba umožňuje, zavolať
registrované funkcie na vzdialenom serveri, ku ktorému máme prístup iba cez internet.

Na funkcie ktoré je takto možné volať sú kladené nasledovné nároky:

\begin{itemize}
  \item Funkcia musí byť metódou exportovaného typu
  \item Funkcia musí byť exportovaná
  \item Funkcia musí akceptovať 2 argumenty, oba exportovaného alebo vstavaného typu
  \item Druhý argument funkcie musí byť smerník
  \item Návratový typ funkcie musí byť typ \texttt{error}
\end{itemize}

Modul RPC podporuje niekoľko protokolov na prenos údajov, vrátane možnosti implementovať
vlastný. Medzi najpoužívanejľie patrí HTTP, prípadne TCP s použitím gob kódovania údajov.
Ďalší rozšírený formát na prenos údajov je JSON.

\subsection{Serializačný formát gob}
\label{sec:golang:gob}

Gob je formát na prenos binárnych údajov cez sieť. Používa sa napríklad
pri RPC službe na prenos objektov jazyka Go. Je samo-popisujúci, každý prvok obsahuje
okrem údajov aj popis jeho typu. Tento formát vychádza dizajnovo z formátu Protocol
Buffers od spoločnosti Google. Viac informácii o serializácii a význame serializovaných
dát je uvedené v \cite{GolangGobs}.

\subsection{SimpleYAML}
\label{sec:golang:simpleyaml}

SimpleYAML\cite{GolangSimpleyaml} je knižnica pre jazyk Go, určená na prácu s
textovým formátom YAML\@. Jedná sa o veľmi jednoduchú knižnicu, ktorá z textovej
reprezentácie vo formáte YAML vytvorí ekvivalentné natívne objekty jazyka Go.

\section{YAML}
\label{sec:yaml}

YAML je dátový serializačný jazyk, ktorý je ale zároveň aj čitateľný pre ľudí, čo
z neho robí ideálny jazyk pre použitie nielen na prenos údajov, ale napríklad aj na
použitie v konfiguračných súboroch.

Od verzie 1.2 je YAML oficiálne spätne kompatibilný s jazykom JSON\cite{YAMLJSONCompat}, čo znamená,
že parser jazyka YAML dokáže parsovať aj akýkoľvek JSON dokument.

Základnými údajovými typmi jazyka sú, podobne ako v JSON: polia, asociatívne polia, reťazce a z nich
vytvorené zložitejšie vnorené štruktúry. Okrem toho jazyk YAML umožňuje používanie spätných
referencií alebo definíciu používateľských typov (závisle od parsera). Všetky vlastnosti, vrátane
syntaxe popisuje voľne dostupná špecifikácia jazyka \cite{YAMLSpec}.
