\chapter{Implement\'{a}cia}
\label{ch:impl}

Táto kapitola sa venuje implementácii aplikácie. Konkrétne popíšeme úlohy
jednotlivých komponentov, detailne vysvetlíme fungovanie a správanie celého
systému, pozrieme sa na zaujímavé časti kódu a rozoberieme rôzne problémy,
na ktoré sme počas implementácie narazili.

\section{Master}
\label{sec:master}

Master komponent je hlavná, používateľská, časť aplikácie. Implementovali sme ju
v jazyku Go. Pri spustení aplikácia kontaktuje Slave komponenty, ktoré sú
určené nastaveniami, od každého si vyžiada informácie potrebné pre distribúciu
úloh a takto inicializovaná začne so spracovávaním cieľov zadaných používateľom.

V rámci inicializácie taktiež spúšťame lokálny Slave komponent, ktorého úlohou je
spracovávanie úloh na lokálnom počítači.

V nasledujúcich častiach postupne popíšeme jednotlivé časti a funkcionalitu Master
komponentu.

\subsection{Spracov\'{a}vanie BUILD s\'{u}borov}
\label{sec:master:buildfile}

BUILD súbory obsahujú popis projektu z pohľadu nášho systému. Súbory štandardne
pomenúvame \texttt{build.yaml} a ich obsah je vo formáte YAML\@. Na spracovanie
formátu YAML v jazyku Go sme sa rozhodli použiť knižnicu
\texttt{simpleyaml}\footnote{\url{https://github.com/smallfish/simpleyaml}}.
Táto knižnica prečíta text vo formáte YAML a umožní s údajmi pracovať pomocou
štandardných údajových štruktúr jazyka Go, ako je napríklad integer, string,
pole alebo mapa.

Listing~\ref{lst:client:parsefile} obsahuje zdrojový kód funkcie ParseFile, ktorá
s použitím knižnice simpleyaml spracúva BUILD súbory a vytvára z nich objekty
reprezentujúce ciele. Formát BUILD súborov, aj s príkladmi uvádzame v časti~\ref{sec:buildfiles}.

\begin{listing}[H]
  \inputminted[frame=lines,framesep=2mm,linenos,fontsize=\scriptsize,firstline=65,lastline=96]{go}{/home/pepol/src/imterra/forge/client/target/parse.go}
  \caption[Funkcia ParseFile]{Funkcia ParseFile, ktorá spracúva BUILD súbory}
  \label{lst:client:parsefile}
\end{listing}

\subsection{Reprezentácia cieľov, rozhranie Target}
\label{sec:master:target}

\begin{listing}[H]
  \inputminted[frame=lines,framesep=2mm,linenos,fontsize=\scriptsize,firstline=8,lastline=14]{go}{/home/pepol/src/imterra/forge/client/target/target.go}
  \caption[Rozhranie Target]{Rozhranie popisujúce objekt reprezentujúci cieľ kompilácie}
  \label{lst:client:target}
\end{listing}

Každý typ cieľa reprezentujeme ako triedu implementujúcu rozhranie typu Target.
Jeho definícia je uvedená v listingu~\ref{lst:client:target}. Postupne popíšeme
jednotlivé metódy.

\subsubsection{GetName}
Metóda vracajúca názov daného cieľa ako je uvedený v BUILD súbore.

\subsubsection{GetSources}
Táto metóda vráti zoznam zdrojových súborov daného cieľa. V prípade neexistenie
takýchto súborov, metóda vráti prázdne pole. Zdrojové súbory sú v BUILD súbore
špecifikované ako zoznam \texttt{sources}.

Medzi zdrojové súbory patria hlavne zdrojové súbory programu v určitom
programovacom jazyku, ktoré je potrebné skompilovať. Sú to najčastejšie textové
súbory so špecifickou príponou.

\subsubsection{GetResources}
Metóda vracajúca zoznam súborov prostriedkov. Funkcionalita je rovnaká ako v
prípade metódy GetSources. Súbory prostriedkov sú špecifikované ako zoznam
\texttt{resources}.

Medzi súbory prostriedkov patria súbory, ktoré je potrebne spracovať iným spôsobom
ako zdrojový kód programu. Ako príklad uvádzame ikony, obrázky alebo videá v
prípade počítačových hier.

\subsubsection{GetDependencies}
Táto metóda vráti zoznam cieľov, na ktorých aktuálny cieľ závisí. Každý člen
tohto zoznamu implementuje rozhranie Target. Pomocou tejto metódy pristupujeme
k vzniknutému grafu závislostí.

\subsubsection{GetOutputFile}
Táto metóda vráti referenciu na vygenerovaný súbor, objekt typu File. Pomocou tohto
objektu pristupujeme ku bipartitnému grafu súborov a akcií.

\subsection{Trieda LibCTarget}
\label{sec:master:libctarget}

Trieda LibCTarget implementuje cieľ typu \verb!lib_c!, ktorý sa používa na
kompiláciu zdrojových kódov jazyku C, ktoré potom môžu využívať rôzne aplikácie.

Trieda LibCTarget vygeneruje staticky linkovanú knižnicu obsahujúcu skompilované
zdrojové súbory. Takto vygenerovanú knižnicu môžeme použiť vo fáze linkovania
akejkoľvek aplikácie.

\begin{figure}[h]
  \centerline{\includegraphics[width=0.6\textwidth]{images/libctarget}}
  \caption[Graf akcií pre typ LibCTarget]{Ukážka grafu akcií vytvoreného pre typ
  LibCTarget}
  \label{img:libctarget}
\end{figure}


Ukážka vytvoreného grafu akcií pre OS Linux je znázornená na obrázku~\ref{img:libctarget},
skladá sa z dvoch akcií: V prvej fáze je každý zdrojový súbor skompilovaný
do objektového súboru jazyka C\footnote{kompilátor jazyka C sa štandardne označuje
pomocou písmen CC, z anglického C Compiler} a následne sú všetky takto skompilované súbory
vložené do archívu vytvoreného programom ar(1)\footnote{\url{http://linux.die.net/man/1/ar}}.

\subsection{Trieda AppCTarget}
\label{sec:master:appctarget}

Táto trieda reprezentuje cieľ typu \verb!app_c!, ktorý sa používa na špecifikáciu
spustiteľného programu / aplikácie napísanej v jazyku C.

Trieda AppCTarget vygeneruje staticky linkovaný spustiteľný súbor obsahujúci
skompilované zdrojové kódy, zlinkované s knižnicami uvedenými ako závislosťami.

\begin{figure}[h]
  \centerline{\includegraphics[width=0.6\textwidth]{images/appctarget}}
  \caption[Graf akcií pre typ AppCTarget]{Ukážka grafu akcií pre typ AppCTarget,
  súbor c.a je výstupom cieľa \texttt{c} typu LibCTarget}
  \label{img:appctarget}
\end{figure}

Na obrázku~\ref{img:appctarget} sa nachádza ukážka grafu akcií pre AppCTarget
v prostredí OS Linux. Tento graf je tvorený akciami kompilácie (pre zdrojové
súbory) a linkovania\footnote{štandardný linker v OS UNIX a Linux má názov ld,
z ktorého vzniklo označenie LD} výsledných objektov s knižnicami.

\subsection{Prevod grafu cie\v{l}ov do grafu akci\'{i}}
\label{sec:master:target2action}

Ako uvádzame v časti~\ref{sec:master:target}, metóda GetOutputFile rozhrania Target
je zodpovedná za vytvorenie správneho grafu akcií pre daný cieľ. Graf akcií je
bipartitný graf zlozžený z objektov rozhrania Action (akcie, úlohy) a objektov
rozhrania File (zdrojové alebo vygenerované súbory).

Nasleduje popis funkcionality metódy pre jednotlivé triedy:

\subsubsection{LibCTarget}

\begin{listing}[h]
  \inputminted[frame=lines,framesep=2mm,linenos,fontsize=\scriptsize,firstline=23,lastline=61]{go}{/home/pepol/src/imterra/forge/client/actions/util.go}
  \caption[Pomocná metóda MakeCObjects]{Implementácia metódy generujúcej akcie
  kompilácie súborov v jazyku C}
  \label{lst:target2action:makecobjs}
\end{listing}

V pomocnej metóde MakeCObjects, ktorú sme uviedli v listingu~\ref{lst:target2action:makecobjs},
prechádzame zoznam mien zdrojových súborov a z každého vytvoríme dvojicu objektov
typu File, jeden reprezentujúci zdrojový súbor a druhý reprezentujúci vygenerovaný
objektový súbor. Následne pre každú takúto dvojicu zdrojového a z neho vygenerovaného
súboru vytvoríme akciu typu CompileCAction (rozhrania Action), ktorej vstupným
súborom bude zdrojový súbor a výstupným vygenerovaný objektový.

Následne vytvoríme súbor reprezentujúci vytváranú knižnicu ako celok (typu File),
ktorý bude výstupným súborom prislúchajúcej akcie typu ArLinkAction (rozhrania Action).
V predchádzajúcom kroku vygenerované objektové súbory použijeme ako vstupné
súbory pre túto akciu. Implementačné detaily uvedieme v listingu~\ref{lst:target2action:libc}.

\begin{listing}[H]
  \inputminted[frame=lines,framesep=2mm,linenos,fontsize=\scriptsize,firstline=48,lastline=74]{go}{/home/pepol/src/imterra/forge/client/target/target.go}
  \caption{Metóda GetOutputFile triedy LibCTarget}
  \label{lst:target2action:libc}
\end{listing}

\subsubsection{AppCTarget}
Metóda sa na zdrojových súboroch správa rovnako ako v triede LibCTarget. Po
vytvorení takejto knižnice prejdeme rekurzívne závislosti a ich výstupné
súbory, spolu s práve vytvorenou knižnicou predáme ako vstupné súbory
akcii LinkAction (rozhrania Action), ktorej výstupným súborom (typu File)
je požadovaná spustiteľná aplikácia. Implementačné detaily sa nachádzajú v
listingu~\ref{lst:target2action:appc}.

\begin{listing}[H]
  \inputminted[frame=lines,framesep=2mm,linenos,fontsize=\scriptsize,firstline=92,lastline=131]{go}{/home/pepol/src/imterra/forge/client/target/target.go}
  \caption{Metóda GetOutputFile triedy AppCTarget}
  \label{lst:target2action:appc}
\end{listing}

\subsection{Distrib\'{u}cia akci\'{i}}
\label{sec:master:distribution}

Na účely vykonania všetkých akcií a získania výstupneho súboru pre každý cieľ sme
implementovali funkciu MakeFile. Táto funkcia sa spúšťa asynchrónne s informáciami
o tom aký súbor sa má získať, aké nastavenia sa pritom majú použiť. Súčasťou je
aj určenie kanálu (dátový typ chan), ktorý sa použije na informovanie volajúceho
o úspešnom získaní súboru. V prípade akejkoľvek chyby funkcia signalizuje chybu,
oznámi ju používateľovi a spôsobi ukončenie behu aplikácie.

\subsubsection{Získanie vstupných súborov}

Funkcia funguje rekurzívne, v prípade, že súbor sa dá získať spustením akcie, ju
najprv asynchrónne zavoláme na vstupné súbory tejto akcie a počkáme, kým všetky
úspešne skončia.

\subsubsection{Zabránenie opakovanej kompilácii}

Kedže chceme zabrániť opakovanej kompilácii, pokiaľ sa vstupné súbory nezmenili,
v ďalšom kroku skontrolujeme novosť súborov. Za týmto účelom máme v koreňovom
adresári skrytú zložku \texttt{.metadata}, ktorá obsahuje súborovú štruktúru
ekvivalentnú s projektami. Každý súbor, nazvaný rovnako ako jeho ekvivalent s ktorým
pracujeme obsahuje iba textovú informáciu, SHA512 kontrolný súčet obsahu súboru
počas jeho predchádzajúceho použitia. Tieto informácie zapisujeme po ukončení vykonávania
všetkých akcií, tesne pred ukončením behu aplikácie. Takto získaný kontrolný súčet
porovnáme s kontrolným súčtom existujúceho súboru a v prípade, že sa tieto líšia,
vykonáme akciu. V prípade neexistencie súboru je akcia vykonaná tiež.

\subsubsection{Výber Slave komponentu na vykonanie akcie}

% TODO describe worker selection

\subsubsection{Vykonanie akcie}

Master komponent pred samotným vykonaním akcie musí zabezpečiť, aby volaný Slave
mal prístup ku všetkým súborom, ktoré akcia potrebuje na úspešné vykonanie. To
zabezpečujeme pomocou metódy SendFile RPC služby File, ktorú popisujeme v časti~\ref{sec:slave:rpc:file}.
V prípade používania lokálne bežiaceho Slave komponentu, je táto časť vynechaná.

Po úspešnom odoslaní všetkých súborov je akcia vykonaná zavolaním ekvivalentnej metódy
RPC služby Task na komponente. Túto službu, vrátane metód pre jednotlivé akcie popisujeme
v časti~\ref{sec:slave:rpc:task}.

V prípade, že volaný komponent nebeží lokálne, následne vyžiadame obsah vytvoreného
súboru pomocou metódy RecvFile služby File. Následne tento obsah zapíšeme do nášho
pracovného priečinka, informujeme volajúceho o úspešnom skončení a ukončíme beh funkcie.

\subsubsection{Problémy a riešenia}

Najväčším problémom s ktorým sme sa stretli bolo zabezpečenie, aby akcie neboli
volané viac krát, v prípade, že niekoľko rôznych súborov záleží na rovnakom
súbore. Kedže systém je implementovaný ako vysoko paralelný, vznikol tu potenciálny
kritický súbeh (race condition), kedy by sa počas vykonávania akcie na Slave komponente
spustilo vykonávanie akcie v inom vlákne.

Na zabránenie tomuto problému sme zjednotili objekty súborov pre celú aplikáciu a prístup
k nim synchronizovali pomocou zamykacej techniky semafórov.

Na zabezpečenie efektívnejšieho spracovania sme tiež v rámci kritickej sekcie, po spustení
vykonávania akcie zabezpečili, aby ďalšie volania funkcie MakeFile považovali vytvorený
súbor za zdrojový a teda sa nepokúšali vykonávať akcie spojené s jeho získaním. V údajovom
type File sme toto dosiahli nastavením smerníku na akciu na \texttt{nil}, čo je v jazyku
Go reprezentácia prázdneho smerníku.

\section{Slave}
\label{sec:slave}

Slave komponent je určený na vykonávanie úloh. Implementujeme ho v jazyku Go
ako RPC\footnote{Remote Procedure Call - vzdialené volanie procedúr} server.
V rámci tohto serveru poskytujeme niekoľko implementovaných RPC služieb,
z ktorých každá implementuje pomocou svojich procedúr určitú časť funkcionality.

V nasledujúcich častiach si popíšeme jednotlivé služby a procedúry ktoré
poskytujú.

\subsection{RPC služba Task}
    \label{sec:slave:rpc:task}

Služba Task poskytuje procedúry na vykonávanie úloh priamo súvisiacich so
splnením cieľa. Tieto úlohy su ekvivalentné akciám na strane Master komponentu.

\begin{listing}[H]
  \inputminted[frame=lines,framesep=2mm,linenos,fontsize=\scriptsize,firstline=8,lastline=12]{go}{/home/pepol/src/imterra/forge/proto/proto.go}
  \caption[Argumenty služby Task]{Štruktúra špecifikujúca argumenty procedúr RPC služby Task}
  \label{lst:rpc:action:args}
\end{listing}

\begin{listing}[H]
  \inputminted[frame=lines,framesep=2mm,linenos,fontsize=\scriptsize,firstline=14,lastline=17]{go}{/home/pepol/src/imterra/forge/proto/proto.go}
  \caption[Návratová hodnota služby Task]{Štruktúra špecifikujúca návratovú hodnotu procedúr RPC služby Task}
  \label{lst:rpc:action:response}
\end{listing}

Každá z týchto procedúr dostáva ako argumenty meno akcie, zoznam vstupných súborov
s cestou relatívnou ku koreňovému priečinku systému, ich SHA512 kontrolným súčtom a informáciu, či volajúci
vyžaduje poslanie výsledného súboru. Implementáciu uvádzame v listingu~\ref{lst:rpc:action:args}.

Výsledkom volania procedúry je výstup, ktorý vznikol počas jej vykonávania a,
pokiaľ volajúci o to požiadal, obsah vzniknutého súboru. Implementáciu uvádzame
v listingu~\ref{lst:rpc:action:response}.

Pri spúšťaní Slave komponentu umožňujeme používateľovi zvoliť, koľko úloh súčasne sa
môže vykonávať pomocou argumentu \texttt{jobs}. Ako prednastavenú hodnotu sme zvolili počet
CPU ktoré sa na danom počítači nachádzajú. Tento počet ovplyvňuje počet súčasne
aktívnych procedúr služby Task, ktoré na zabezpečenie používajú semafór implementovaný
pomocou atribútu objektu služby. Funkcionalitu semafóru sme dosiahli pomocou
dátového typu channel, ktorému určujeme veľkosť na základe hodnoty argumentu \texttt{jobs}.
Zároveň u každej procedúry tejto služby vyžadujeme, aby pri spustení zapísala do semafóru
údaje, akcia ktorá v jazyku Go blokuje, ak je objekt naplnený, a po skončení vykonávania,
úspešnom ako aj neúspešnom, hodnotu zo semafóru prečítala a tým uvoľnila miesto pre
vykonávanie ďalších procedúr.

\begin{listing}[H]
  \inputminted[frame=lines,framesep=2mm,linenos,fontsize=\scriptsize,firstline=9,lastline=22]{go}{/home/pepol/src/imterra/forge/worker/tasks/util.go}
  \caption{Funkcia na prípravu argumentov pre volanie príkazu}
  \label{lst:rpc:processinputs}
\end{listing}

\begin{listing}[H]
  \inputminted[frame=lines,framesep=2mm,linenos,fontsize=\scriptsize,firstline=24,lastline=40]{go}{/home/pepol/src/imterra/forge/worker/tasks/util.go}
  \caption{Funkcia na prípravu odpovede služby Task}
  \label{lst:rpc:prepareresp}
\end{listing}

V nasledujúcich častiach popisujeme jednotlivé procedúry služby Task. V týchto
procedúrach sú činnosti ako príprava odpovede na odoslanie (listing~\ref{lst:rpc:prepareresp})
alebo príprava argumentov pre volanie externého príkazu (listing~\ref{lst:rpc:processinputs})
často opakované, preto sme sa rozhodli ich implementácie uviesť samostatne.

\subsubsection{CompileC}

Táto procedúra vykonáva kompiláciu C súboru do objektového súboru. Vzhľadom na to,
že táto akcia vyžaduje iba 1 zdrojový súbor, rozhodli sme sa signalizovať chybu
kedykoľvek na vstupe dostaneme viac súborov.

\begin{listing}[H]
  \inputminted[frame=lines,framesep=2mm,linenos,fontsize=\scriptsize,firstline=20,lastline=47]{go}{/home/pepol/src/imterra/forge/worker/tasks/tasks.go}
  \caption[Implementácia kompilácie jazyka C]{Implementácia úlohy kompilácie programovacieho jazyka C pre OS Linux}
  \label{lst:rpc:compilec}
\end{listing}

V listingu~\ref{lst:rpc:compilec} sme ukázali implementáciu procedúry pre OS Linux
a kompilátor GNU GCC\@. Táto procedúra volá externý príkaz s predpripravenými parametrami,
ktorého výstup posunie v návratovej hodnote volajúcemu.

\subsubsection{ArLink}

Táto procedúra vytvorí staticky linkovateľnú knižnicu. V OS Linux sa jedná o archív
obsahujúci jednotlivé objektové súbory, ktorý je vytvorený programom ar, podľa ktorého
sme pomenovali aj procedúru. Jej implementáciu pre OS Linux uvádzame v listingu~\ref{lst:rpc:arlink}.
Rovnako ako v prípade CompileC procedúry sa jedná o volanie externého príkazu.

\begin{listing}[H]
  \inputminted[frame=lines,framesep=2mm,linenos,fontsize=\scriptsize,firstline=49,lastline=72]{go}{/home/pepol/src/imterra/forge/worker/tasks/tasks.go}
  \caption{Implementácia staticky linkovanej knižnice pre OS Linux}
  \label{lst:rpc:arlink}
\end{listing}

\subsubsection{LdLink}

Procedúra LdLink, nazvaná podľa programu ld(1) určeného na linkovanie programov,
vytvára výslednú aplikáciu zlinkovaním statických knižníc uvedených ako závislosti
cieľa a objektových súborov, ktoré vzniknú kompiláciou zo zdrojových kódov programu.

\begin{listing}[H]
  \inputminted[frame=lines,framesep=2mm,linenos,fontsize=\scriptsize,firstline=74,lastline=97]{go}{/home/pepol/src/imterra/forge/worker/tasks/tasks.go}
  \caption{Implementácia linkovania aplikácie v OS Linux}
  \label{lst:rpc:ldlink}
\end{listing}

V OS Linux implementácia volá GNU GCC\@, ako uvádzame v listingu~\ref{lst:rpc:ldlink}, ktorý
poskytuje zjednodušené linkovanie, vďaka množstvu predpripravených nastavení pre program
ld.

\subsection{RPC služba File}
\label{sec:slave:rpc:file}

Služba File poskytuje možnosti prenosu súborov medzi Slave a Master komponentmi.
V rámci tejto služby implementujeme nasledovné procedúry:

\begin{listing}[h]
  \inputminted[frame=lines,framesep=2mm,linenos,fontsize=\scriptsize,firstline=7,lastline=21]{go}{/home/pepol/src/imterra/forge/proto/files.go}
  \caption{Štruktúry požiadaviek a odpovedí služby File}
  \label{lst:rpc:fileargs}
\end{listing}


\subsubsection{SendFile}

Jedná sa o požiadavku volajúceho na obsah súboru nachádzajúceho sa na serveri.
Súčasťou požiadavky je meno súboru a jeho kontrolný súčet algoritmom SHA512.
Odpoveď servera obsahuje meno súboru a jeho obsah, vrátane metadát, ktoré, okrem
iného, popisujú prístupové práva k súboru. Tieto informácie vyžadujeme za účelom
zachovania napr.\ práv na spustenie aplikácie.

Formát požiadavky aj odpovede je opísany v listingu~\ref{lst:rpc:fileargs}.
Požiadavku reprezentuje štruktúra FileRequest a odpoveď štruktúra File.

V prípade, že by súbor neexistoval, alebo nebol prístupný, procedúra signalizuje chybu.

\subsubsection{RecvFile}

Procedúra na odoslanie súboru serveru. Požiadavkou je popis súboru, štruktúra
File v listingu~\ref{lst:rpc:fileargs}. V prípade neexistencie adresára v ktorom
sa má súbor umiestniť, sa ho procedúra pokúsi vytvoriť, následne zapíše súbor
na disk. V prípade akýchkoľvek problémov signalizuje chybu. Výsledok procedúry
popisuje štruktúra FileResponse v listingu~\ref{lst:rpc:fileargs}, ktorá okrem mena
súboru a kontrolného súčtu údajov obsahuje tiež informáciu, či táto požiadavka
prepísala existujúci súbor.

\subsection{RPC služba Util}
\label{sec:slave:rpc:util}

Služba Util je určená na poskytovanie rôznych pomocných procedúr, najmä procedúr
uľahčujúcich distribúciu úloh. V rámci tejto služby poskytujeme nasledovné
procedúry:

\subsubsection{NumTasks}

Jednoduchá procedúra, pomocou ktorej sprístupňujeme informáciu o maximálnom
povolenom množstve simultánne bežiacich úloh na komponente. Túto informáciu
využíva Master na zefektívnenie distribúcie úloh, ako sme opísali v~\ref{sec:master:distribution}.

Vzhľadom na limitácie RPC protokolu, procedúra vyžaduje aspoň jeden argument. Rozhodli
sme sa preto použiť najzákladnejší typ integer, ktorého hodnota je ignorovaná. Návratovou
hodnotou procedúry je taktiež integer, maximálny počet úloh, ako je definovaný pri
spustení Slave komponentu.
